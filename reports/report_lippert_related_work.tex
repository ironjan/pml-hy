\cite{parkendd} is a similar project which used the collected data to predict parking space availability. This project focussed on the car park ``Centrum Galerie''. \cite{parkendd} used the following features: week of year, day of week, time of day, sunday openings and the number of workdays until the next public holiday. Sunday openings are used as in Germany, shops are normally closed on Sundays. Sunday openings therefore influence how many people go to the shopping center and therefore the availability of parking space. In \cite{parkendd}, different algorithms are compared and the predictions are evaluated.

\cite{Rajabioun2013} propose parking guiding and
information system which also includes the prediction of available parking space. The proposed algorithm uses a probabilistic model based on historical data. Among others, the used features also include time of day and day of week. At last, \cite{Rajabioun2013} estimates the mean-error on predictions. They mention that predictions for 10 minutes in the future lead to a 1.2\% error on average while predictions for 40 minutes in the future lead to a 2.8\% error on average.

\cite{Zheng2015} compared three different feature sets and three different machine learning algortihms -- regression tree, neural network, and support vector regression -- with respect to their performance. Based on data from the cities of Melbourne and San Francisco they conclude that the regression tree provides the best predictions when used with a feature set containing day of week and time of day.  