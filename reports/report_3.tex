\documentclass[journal,10pt]{IEEEtran}
\usepackage{hyperref}
\usepackage{todonotes}
\usepackage{amssymb}
\usepackage{amsmath}
\usepackage{mathtools}
\usepackage{graphicx}

\newcommand{\subtask}[1]{\begin{quote}\textbf{#1}\end{quote}}
\newcommand{\hod}{h}
\newcommand{\moh}{m}
\newcommand{\dow}{d}
\newcommand{\dom}{d_m}
\newcommand{\wom}{w_m}
\newcommand{\woy}{w_y}
\newcommand{\yyy}{a}

\newcommand{\IN}{\mathbb{N}}
\newcommand{\defeq}{\coloneqq}

\author{Jan Lippert \(<\)\href{mailt:ljan@mail.upb.de}{ljan@mail.upb.de}\(>\)}
\date{\today}

\begin{document}

\title{Predicting Parking Space Availability in Paderborn\\
Technology Selection}
\maketitle


\section{Introduction}

In urban areas, space and as such parking space is limited. Car owners often have to drive around to available parking spots. This leads to an unnecessary waste of time and additional air pollution. 

According to multiple studies, the availability of parking data reduces the search time for a parking spot \cite{Asakura1994}\cite{Caicedo2010228}. However, most systems only provide the current number free parking spots. In rush hours, this information can quickly get outdated and lead to driving around to find available parking spaces.

This report will describe how such a system could be implemented. Section \ref{sec:architecture} will show how the overall architecture of the system will look like. Section \ref{sec:frameworks} will compare the different frameworks that were considered for the implementation. 


\section{Architecture}\label{sec:architecture}
\begin{itemize}
  \item Multiple modules
  \item crawling, preprocessing, training, prediciton, evaluation
\end{itemize}

\section{Frameworks}\label{sec:frameworks}

\begin{itemize}
  \item RoR, Python, Play!
  \item 
\end{itemize}


\bibliographystyle{ieeetr}  
\bibliography{report_2}

\end{document}
