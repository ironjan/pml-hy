\documentclass[journal,10pt]{IEEEtran}
\usepackage{hyperref}
\usepackage{todonotes}
\usepackage{amssymb}
\usepackage{amsmath}
\usepackage{mathtools}
\usepackage{graphicx}

\title{Predicting Parking Space Availability in Paderborn\\
Other notes}
\author{Jan Lippert \(<\)\href{mailt:ljan@mail.upb.de}{ljan@mail.upb.de}\(>\)}
\date{\today}

\newcommand{\subtask}[1]{\begin{quote}\textbf{#1}\end{quote}}
\newcommand{\hod}{h}
\newcommand{\moh}{m}
\newcommand{\dow}{d}
\newcommand{\dom}{d_m}
\newcommand{\wom}{w_m}
\newcommand{\woy}{w_y}
\newcommand{\yyy}{a}

\newcommand{\IN}{\mathbb{N}}
\newcommand{\defeq}{\coloneqq}

\begin{document}
\maketitle


\section{Heroku Free Dynos}
After collecting data for some time, I did some correlation analysis on the data. I had to notice that all entries were labelled as ``\(\wom = 1\)'' despite the app being put online more than a week ago. Further investigation showed that the data was only available for 3 different days. 

The reason for this failure of data collection was in the uninformed use of the Heroku Free package. Free dynos will sleep after 30\todo{60? minutes} minutes of inactivity. After this was noticed, \url{http://kaffeine.herokuapp.com/} was set up to keep the application alive. 

Free dynos are required to sleep for 6 hours a day. Since the ``Libori-Galerie'' is closed from 2:00 am to 8:00 am we will align the sleeping schedule of our application accordingly. 


\bibliographystyle{ieeetr}  
\bibliography{report_notes}

\end{document}
