\documentclass[journal,10pt]{IEEEtran}
\usepackage{hyperref}

\title{Predicting Free Parking Spots in Paderborn}
\author{Jan Lippert \(<\)\href{mailt:ljan@mail.upb.de}{ljan@mail.upb.de}\(>\)}
\date{\today}


\begin{document}
\maketitle

\begin{abstract}
This is the paper's abstract \ldots

Please submit a one-page PDF report, 

\begin{itemize}
  \item explaining your topic and why you've chosen it
  \item Try to position your topic amongst known research.
  \item Explain where you're going to get your data from and\ldots
  \item what you're going to predict.
\end{itemize}

Don't go into details about implementation yet.
\end{abstract}

\section{Result to Predict}
I'm originally from Paderborn and when using the car, I often searched for free parking spaces in the city managed parking lots. It would have been nice to know in advance where to park my car. And as such, I want to use machine learning to predict the number of free parking spots on the parking areas in Paderborn.

Previously, I encountered the project \url{https://parkendd.de} which shows the current free parking spots in different cities and used the acquired data to predict the usage of the covered car park ``Centrum Galerie''\footnote{\url{http://mechlab-engineering.de/2015/03/vorhersage-der-parkhausbelegung-mit-offenen-daten/}}. The results presented on this page show that machine learning can be used to predict the number of free parking spots.

\section{Data Sources}\label{data sources}
There are many possible data sources in addition to the actual parking data. Taking all possible influence factors into account is most likely not be possible. Therefore we will focus on the parking data and optionally extend the model with event and weather data.

\paragraph{Parking Usage Live Data}
The main source of data for this project will be the homepage of the ASP Paderborn which lists the city managed parking areas and the number of currently free parking spots. There is no API known for the displayed data. 

The data is displayed on \url{https://www.paderborn.de/microsite/asp/parken_in_der_city/freie_Parkplaetze_neu.php}; the displayed table is actually a very minimalistic website located at \url{https://www4.paderborn.de/ParkInfoASP/default.aspx}. Scraping this website should be very easy.

\paragraph{Event Data (Optional)}
Another interesting data source could be public holidays and local events. In some cases, these events influence the number of free parking spots directly: the Libori fair is located on the parking area ``Le Mans Wall'' and the ``Lunapark'' is located on ``Maspernplatz''. 

Other events may also cause more people to go to the city center and therefore to more parking space usage. One example includes Sunday openings. In Germany, shops are normally closed on Sunday. Howevery, cities may choose to allow shopping on a few Sundays a year. These Sunday openings most likely lead to a higher parking space usage than on other Sundays.

This data is considered optional as the big events do not happen while the course ``Practical Project in Machine Learning'' takes place. A list of these events can be found on the homepage of the carneval club\footnote{\url{http://www.kirmes-paderborn.de/termine.htm}}. The city's homepage also lists some events\footnote{\url{https://www.paderborn.de/tourismus-kultur/veranstaltungen/veranstaltungshighlights.php}}. Both of these sources would have to be parsed manually as there is no ICS-Feed or similar. 

\paragraph{Weather Data (Optional)}

The current weather may also influence how many people go to the city center and how many of them use a car. Taking weather data into account is an optional task because it is assumed that the weather has only a small influence on the number of free parking spots.


\section{?}

\begin{itemize}
 \item Assumption: parking spot usage depends on time, e.g. many people go shopping in the afternoon
 \item It is assumed that the parking spot usage fluctuates over time depending e.g. on weekdays or similar
 \item Therefore the crawling timestamp is split into its parts (y, m, d, h, M) and extended (day of week, day of month, ...).
\end{itemize}

\section{How do predictions look like?}

The pml-hy tool should be able to answer the following three queries:
\begin{itemize}
 \item How many free parking spots are there in 5 minutes?
 \item How many free parking spots are there in 15 minutes?
 \item How many free parking spots are there in 60 minutes?
\end{itemize}


\bibliographystyle{abbrv}
\bibliography{simple}

\end{document}
