\documentclass[journal,10pt]{IEEEtran}
\usepackage{hyperref}
\usepackage{todonotes}

\title{Predicting Free Parking Spots in Paderborn}
\author{Jan Lippert \(<\)\href{mailt:ljan@mail.upb.de}{ljan@mail.upb.de}\(>\)}
\date{\today}


\begin{document}
\maketitle

\section{Introduction}

In urban areas, space and as such parking space is limited. Car owners often have to drive around to available parking spots. This leads to an unnecessary waste of time and additional air pollution. 

According to multiple studies, the availability of parking data reduces the search time for a parking spot \cite{Asakura1994}\cite{Caicedo2010228}. However, most systems only provide the current number free parking spots. In rush hours, this information can quickly get outdated and lead to driving around to find available parking spaces.

%\cite{parkendd} is a similar project which shows the current free parking spots in different cities. This information is collected from different websites. In addition, \cite{parkendd} collected data over the course of a year and used machine learning to predict the usage of the covered car park ``Centrum Galerie''\footnote{\url{http://mechlab-engineering.de/2015/03/vorhersage-der-parkhausbelegung-mit-offenen-daten/}}. 
%\cite{Rajabioun2013} and \cite{Zheng2015} propose similar systems to predict the availability of parking space to help car owners find a parking spot in advance. 

In Paderborn the situation is similar: live-data is available but no prediction. The proposed project will collect parking data from Paderborn and use machine learning to predict the number of free parking spots.



\section{Related Work}

\cite{parkendd} is a similar project which used the collected data to predict parking space availability. This project focussed on the car park ``Centrum Galerie''. \cite{parkendd} used the following features: week of year, day of week, time of day, sunday openings and the number of workdays until the next public holiday. Sunday openings are used as in Germany, shops are normally closed on Sundays. Sunday openings therefore influence how many people go to the shopping center and therefore the availability of parking space. In \cite{parkendd}, different algorithms are compared and the predictions are evaluated.

\cite{Rajabioun2013} propose parking guiding and
information system which also includes the prediction of available parking space. The proposed algorithm uses a probabilistic model based on historical data. Among others, the used features also include time of day and day of week. At last, \cite{Rajabioun2013} estimates the mean-error on predictions. They mention that predictions for 10 minutes in the future lead to a 1.2\% error on average while predictions for 40 minutes in the future lead to a 2.8\% error on average.

\cite{Zheng2015} compared three different feature sets and three different machine learning algortihms -- regression tree, neural network, and support vector regression -- with respect to their performance. Based on data from the cities of Melbourne and San Francisco they conclude that the regression tree provides the best predictions when used with a feature set containing day of week and time of day.  


\section{Data Sources}\label{data sources}
The main data source is the actual parking data which is available online. In addition there are other possible features that influence parking space availability. Since taking all features into account is not possible, this project will focus on the parking data and optionally extend the model with basic event data.

\paragraph{Parking Usage Live Data}
The homepage of the ASP Paderborn\footnote{City-Managed service for waste management, city cleaning, and parking.} lists the city managed parking areas and the number of currently free parking spots. 
The data is available on \url{https://www.paderborn.de/microsite/asp/parken_in_der_city/freie_Parkplaetze_neu.php}. A more minimal website is available at \url{https://www4.paderborn.de/ParkInfoASP/default.aspx}. 

Since there is no API known, the parking data needs to be scraped from the website.  


\paragraph{Event Data (Optional)}
A second data source could be public holidays and local events. In some cases, these events influence the number of free parking spots directly: the Libori fair is located on the parking area ``Le Mans Wall'' and the ``Lunapark'' is located on ``Maspernplatz''. 

Other events may also cause more people to go to the city center and therefore to more parking space usage. One example includes Sunday openings. In Germany, shops are normally closed on Sunday. Howevery, cities may choose to allow shopping on a few Sundays a year. 

This data is considered optional as the big events do not happen while the course ``Practical Project in Machine Learning'' takes place. A list of these events can be found on the homepage of the carneval club\footnote{\url{http://www.kirmes-paderborn.de/termine.htm}}. The city's homepage also lists some events\footnote{\url{https://www.paderborn.de/tourismus-kultur/veranstaltungen/veranstaltungshighlights.php}}. Both of these sources would have to be parsed manually.


\section{Prediction Goal}
As mentioned in the introduction, this project is about predicting available parking space ahead of time. In concrete terms, the project will first focus on one of the parking areas, e.g. ``Libori-Galerie'', and predict the availabilty on regular intervals for 5, 15, and 60 minutes into the future.


\bibliographystyle{abbrv}
\bibliography{report_1}

\end{document}
