\documentclass[journal,10pt]{IEEEtran}
\usepackage{hyperref}

\title{A Very Simple \LaTeXe{} Template}
\author{
        Vitaly Surazhsky \\
                Department of Computer Science\\
        Technion---Israel Institute of Technology\\
        Technion City, Haifa 32000, \underline{Israel}
            \and
        Yossi Gil\\
        Department of Computer Science\\
        Technion---Israel Institute of Technology\\
        Technion City, Haifa 32000, \underline{Israel}
}
\date{\today}


\begin{document}
\maketitle

\begin{abstract}
This is the paper's abstract \ldots
\end{abstract}

\section{Result to Predict} How many free parking spots are there in Paderborn at what time in the future? And where? This idea was inspired by \url{https://parkendd.de} and can be useful for 

\begin{itemize}
  \item Travel planning
  \item Construction planning (questionable?)
\end{itemize}

\section{Data used for prediction}\label{data sources}
\paragraph{The live data from Paderborn}

\begin{itemize}
  \item Available at \url{https://www.paderborn.de/microsite/asp/parken_in_der_city/freie_Parkplaetze_neu.php}
  \item There is no known API. The data can be crawled more easily from \url{https://www4.paderborn.de/ParkInfoASP/default.aspx}
  \item Shows a list of parking lots in Paderborn and the current number of free spots
  \item We can scrape this data and add timestamps
\end{itemize}

\paragraph{Optional: Weather}
\begin{itemize}
  \item Influence may be very small as the city center can be also reached by public transportation
  \item There may be APIs to get weather data for paderborn
\end{itemize}

\paragraph{Optional: Public Holidays, Important local events}
\begin{itemize}
  \item There are multiple events in Paderborn that influence parking
  \item On Libori (local fair), the parking lot ``Le Mans Wall'' is used as location.
  \item Also, the usage could be influenced by public holidays
  \item Data may be available as ICS feed or could be entered manually
\end{itemize}


\section{How will I predict the result?}

\begin{itemize}
 \item Assumption: parking spot usage depends on time, e.g. many people go shopping in the afternoon
 \item It is assumed that the parking spot usage fluctuates over time depending e.g. on weekdays or similar
 \item Therefore the crawling timestamp is split into its parts (y, m, d, h, M) and extended (day of week, day of month, ...).
\end{itemize}

\section{How do predictions look like?}

The pml-hy tool should be able to answer the following three queries:
\begin{itemize}
 \item How many free parking spots are there in 5 minutes?
 \item How many free parking spots are there in 15 minutes?
 \item How many free parking spots are there in 60 minutes?
\end{itemize}


\bibliographystyle{abbrv}
\bibliography{simple}

\end{document}
