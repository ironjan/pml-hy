\documentclass[journal,10pt]{IEEEtran}
\usepackage{hyperref}
\usepackage{todonotes}

\title{Predicting Parking Space Availability in Paderborn}
\author{Jan Lippert \(<\)\href{mailt:ljan@mail.upb.de}{ljan@mail.upb.de}\(>\)}
\date{\today}

\newcommand{\subtask}[1]{\begin{quote}\textbf{#1}\end{quote}}
\begin{document}
\maketitle

\begin{abstract}
  \begin{itemize}
    \item Describe your data source
    \item Formally (=precisely) describe your hypothesis
    \item Describe your data summarily, does it seem feasible for your idea? Discuss any problem you have identified
    \item Describe your selected model training method and discuss the reasoning behind your selection.
    \item Describe and discuss your ``objective method'', ie. the method by which you will measure the success of your predictor.
  \end{itemize}
\end{abstract}

\section{Introduction}

In urban areas, space and as such parking space is limited. Car owners often have to drive around to available parking spots. This leads to an unnecessary waste of time and additional air pollution. 

According to multiple studies, the availability of parking data reduces the search time for a parking spot \cite{Asakura1994}\cite{Caicedo2010228}. However, most systems only provide the current number free parking spots. In rush hours, this information can quickly get outdated and lead to driving around to find available parking spaces.

%\cite{parkendd} is a similar project which shows the current free parking spots in different cities. This information is collected from different websites. In addition, \cite{parkendd} collected data over the course of a year and used machine learning to predict the usage of the covered car park ``Centrum Galerie''\footnote{\url{http://mechlab-engineering.de/2015/03/vorhersage-der-parkhausbelegung-mit-offenen-daten/}}. 
%\cite{Rajabioun2013} and \cite{Zheng2015} propose similar systems to predict the availability of parking space to help car owners find a parking spot in advance. 

%In Paderborn the situation is similar: live-data is available but no prediction. The proposed project will collect parking data from Paderborn and use machine learning to predict the number of free parking spots.

The goal of this project is to predict the available parking space in advance. Because of different reasons\todo{write in text}, we will focus on one of the parking lots. Data will be collected for all lots and send to the developer or so.


\section{Data Sources}\label{data sources}
The main data source is the actual parking data which is available online. In addition there are other possible features that influence parking space availability. Since taking all features into account is not possible, this project will focus on the parking data and optionally extend the model with basic event data.

\paragraph{Parking Usage Live Data}
The homepage of the ASP Paderborn\footnote{City-Managed service for waste management, city cleaning, and parking.} lists the city managed parking areas and the number of currently free parking spots. 
The data is available on \url{https://www.paderborn.de/microsite/asp/parken_in_der_city/freie_Parkplaetze_neu.php}. A more minimal website is available at \url{https://www4.paderborn.de/ParkInfoASP/default.aspx}. 

The website displays 4 features: name, type, capacity, and available spots\footnote{respectively: Parkstätte, --, Anzahl, Frei}. After crawling, the parking information will be annotated with the crawling time and extend the latter into multiple attributes: hour of day, minute of hour, day of week, day of month, week of month, and week of year.

We will ignore the crawling timestamp and focus on only one of the parking areas. It is also assumed that the capacity does not change over time\todo{can we do this?}. The resulting dataset will therefore have 7 attributes. I only use the 

%\paragraph{Event Data (Optional)}
%A second data source could be public holidays and local events. In some cases, these events influence the number of free parking spots directly: the Libori fair is located on the parking area ``Le Mans Wall'' and the ``Lunapark'' is located on ``Maspernplatz''. 

%Other events may also cause more people to go to the city center and therefore to more parking space usage. One example includes Sunday openings. In Germany, shops are normally closed on Sunday. Howevery, cities may choose to allow shopping on a few Sundays a year. 

%This data is considered optional as the big events do not happen while the course ``Practical Project in Machine Learning'' takes place. A list of these events can be found on the homepage of the carneval club\footnote{\url{http://www.kirmes-paderborn.de/termine.htm}}. The city's homepage also lists some events\footnote{\url{https://www.paderborn.de/tourismus-kultur/veranstaltungen/veranstaltungshighlights.php}}. Both of these sources would have to be parsed manually.

\section{Hypothesis}
\subtask{Formally (=precisely) describe your hypothesis}

\section{Feasibility}
\subtask{Describe your data summarily, does it seem feasible for your idea? Discuss any problem you have identified}

\section{Model}
\subtask{Describe your selected model training method and discuss the reasoning behind your selection.}
\subsection{Training Methods}

\section{Verification}
\subtask{Describe and discuss your ``objective method'', ie. the method by which you will measure the success of your predictor.}


%\section{Prediction Goal}
%As mentioned in the introduction, this project is about predicting available parking space ahead of time. In concrete terms, the project will first focus on one of the parking areas, e.g. ``Libori-Galerie'', and predict the availabilty on regular intervals for 5, 15, and 60 minutes into the future.


\bibliographystyle{abbrv}
\bibliography{report_2}

\end{document}
