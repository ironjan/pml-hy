\subsection{What question does your work answer}
In urban areas, space and as such parking space is limited. Car owners often have to drive around to available parking spots. This leads to an unnecessary waste of time and additional air pollution. 

According to multiple studies, the availability of parking data reduces the search time for a parking spot \cite{Asakura1994}\cite{Caicedo2010228}. However, most systems only provide the current number free parking spots. In rush hours, this information can quickly get outdated and lead to driving around to find available parking spaces.

%\cite{parkendd} is a similar project which shows the current free parking spots in different cities. This information is collected from different websites. In addition, \cite{parkendd} collected data over the course of a year and used machine learning to predict the usage of the covered car park ``Centrum Galerie''\footnote{\url{http://mechlab-engineering.de/2015/03/vorhersage-der-parkhausbelegung-mit-offenen-daten/}}. 
%\cite{Rajabioun2013} and \cite{Zheng2015} propose similar systems to predict the availability of parking space to help car owners find a parking spot in advance. 

In Paderborn the situation is similar: live-data is available but no prediction. The goal of this project is exactly that. Because of the limitation outlined in \ref{sec:challenges}, data-crawling and future predictions will focus on the covered car park ``Libori-Galerie''. This car park is also the most interesting as it is directly connected to a local shopping center. 
%It is open from 06:00 am to 02:00 am.

