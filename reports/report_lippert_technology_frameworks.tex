To implement this project, Ruby on Rails, Python and Scala were considered. I used Ruby on Rails in previous projects and development is quite fast with this framework. However, I could only find a few libraries that deal with machine learning \cite{bigml} \cite{leanpanda}. 
Another choice of language was python. Python has quite a lot pf machine learning libraries and is also used in academics. However, I do not have much previous experience with python.

My final choice was the \href{Play! framework}{https://playframework.com} with Scala. I did use this framework in previous projects and therefore was familiar with setting up background jobs and how to enable web access. Heroku also supports \href{https://devcenter.heroku.com/articles/play-support}{easy deployment for Play! applications}. One important factor of this choice was the type-safety of the Scala language and it's usage in machine learning. 

I chose to use \href{Smile}{http://haifengl.github.io/smile/index.html}, the ``Statistical Machine Intelligence and Learning Engine''. It was easy to use and documentation is quite extensive. In some cases, the API documentation even references the scientific papers the learning algorithm is based on. 
